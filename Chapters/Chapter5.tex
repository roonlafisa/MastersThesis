% Chapter 4

\chapter{Conclusion} % Main chapter title
\label{Chapter4}
\renewcommand{\sectionmark}[1]{\markright{\thesection.\ #1}}
\chead{\rightmark}
\section{Summary of work}

Leaky-wave antennas are known to have wide input impedance bandwidth; however, their emitted beam scans with frequency variations. Chapter \ref{Chapter0} raised a question of whether or not an obliquely radiating fixed-beam leaky-wave antenna can be designed for broadband applications. This question has been addressed through the development of this thesis. A leaky-wave antenna is proposed in this work that eliminates the conventional beam scanning performance while maintaining its wide input impedance. The antenna consists of a graded dielectric superstrate placed on top of a slot array. The superstrate functions as a transition layer to couple the slot-generated leaky wave radiation into free space. 

The radiation mechanism of previously proposed leaky-wave antennas has been presented in Ch. \ref{Chapter1}. Broadband fixed-beam radiation from leaky-wave antennas has received less attention as compared to conventional beam-scanning characteristics. The first fixed beam leaky-wave antenna was implemented by a dielectric lens to generate particularly broadside radiation. Oblique radiation from this antenna can be achieved by simply tilting the antenna structure. However,  the lens structure of the antenna made it electromagnetically large which is unsuitable for planar applications. In another planar approach, fixed-beam radiation was achieved using non-foster circuits. However non-foster circuits have to be implemented using active elements which increases the power consumption of the overall structure. The project reflected in this thesis directed towards the design of a planar fixed-beam broadband leaky-wave antenna.

As presented in Ch. \ref{Chapter1}, an infinitely long slot aperture located at the interface of two different dielectric half-space produces leaky-wave mode along the slot-line. The propagating mode generated leaky wave radiation which, in contrast to conventional one, remained constant with the change of operating frequency. The properties of the generated radiation was characterized as a function of different slot-width and dielectric materials. A leaky-wave antenna was designed with a simple 1D graded index material superstrate. A 1D dielectric distribution was not sufficient enough to provide an agreeable broadband antenna performance.

%In order to achieve an improved wide-band performance from the antenna, a structure with inhomogeneous transition layer was pursued. The preliminary design methodology is described in Ch. \ref{Chapter5}, where a dielectric slab with linear gradient was applied. However, the initial results of the simulated antenna suggested unsatisfactory performance. The radiated beam from the structure had inconsistent direction of radiation, higher sidelobes and low directivity over the operating frequency range. Therefore, the research was directed into a new route though transformation electromagnetics. The transition layer was designed by transforming a convex uniform dielectric medium into a rectangular inhomogeneous domain. The material property of the superstrate was achieved through quasi-conformal transformation. The superstrate, if designed through a straightforward single-step quasi conformal transformation, lead to unexpected behavior from the antenna. As addressed in Ch. \ref{Chapter2}, the unusual radiation pattern emerges due to existing phase-discrepancies at the radiating interface of the superstrate. The phase-discrepancies arise during coordinate stretching of the transformation process. Most illustrations of transformation optics in the literature yields negligible mismatch at the interface in the transformed medium. For the fixed-beam leaky-wave antenna, higher mismatch at the interface creates significant differences between the initial domain and the optically transformed media, especially for oblique waves. Chapter \ref{Chapter2} attempts to address the issue. Based on transformation electromagnetics and coordinate-mapping, a method is proposed that leads to a target phase profile at the air-dielectric interface. The overall process was denoted as the geometric compensation technique. The development of this straightforward, but powerful geometric compensation technique to improve radiation patter of an transformation optics-based antennas has been crucial for the progress of this thesis.

The method was implemented on the rectangular superstrate to design the desired fixed-beam wide-band leaky-wave antenna. The design process is presented in Ch. \ref{Chapter3}. Simulation results presented in the same chapter demonstrates how the antenna radiation improves by applying the proposed technique. The slot-line leaky-wave antenna that is capable to radiate at a fixed-angle, which is designed for a bandwidth of $123\%$. As compared to existing antennas, the proposed design consists of a slot-line array placed underneath an inhomogeneous superstrate, which is purely a passive device. 

To date, a number of transformation electromagnetics-based antenna has been proposed. The geometric compensation technique presented in this thesis opens the door for further improvement in antenna design inspired by transformation electromagnetics, in particular, lens antennas.  In addition, the method can be applied into flat lenses where oblique radiation is required. 

%\section{General Remarks}
%bla bla
\section{Future directions}
Future directions of this research include improving the existing antenna design, fabricating and testing the prototype. The major challenge in the implementation of the proposed antenna is the fabrication of graded dielectric materials. Although the analysis of radiation pattern presented in this thesis is done for an upper limit of $2f_0$, the antenna is expected to operate at even higher frequencies. It is anticipated the proposed antenna to be very useful for fixed-beam applications. Fabrication of graded-index inhomogeneous medium has been studied since 1900s for weak gradients. In 1969, Pearson utilized diffusive ion exchange technique to report a fabrication technique for gradient-index materials for image relays using glass rods \cite{pearson1969}. However, the method was not applicable for waveguide applications due to high absorption and scattering losses. The technique was extensively used for manufacturing graded-index dielectrics in the 1970s. In 1975, Martin demonstrated a fabrication technique that applied to waveguide applications \cite{martin1974}. In later years, gradient-index materials have been fabricated using co-evaporation technique, physical and chemical vapor deposition, pulsed laser deposition, glancing angle deposition (GLAD) techniques \cite{shvartsburg2013}. These techniques are highly applicable for specific applications and sensitive to particular frequency bands and angle of incidence of wave. 

For millimeter wave applications, the additive manufacturing process that employs denser dielectric powders of higher permittivity to merge into a sheet of less denser substrate \cite{Roper2014}. Variation of the density of the powder creates a slab of spatially graded permittivity. The density of the powder dispensed determines the effective relative permittivity of the composite. The additive manufacturing process is suitable for microwave and terahertz frequencies. In order to test the proposed antenna, the graded-dielectric slab requires to be fabricated considering the limitations of the available facilities. Additive manufacturing process appears to be an appropriate solution. However, the technique requires 3D printing deposition gun with control techniques and high-temperature furnace \cite{Roper2014}, which is difficult to set-up. As an alternative, the slab could be fabricated by drilling holes into a denser dielectric slab. The highest refractive index of the slab is less than 3 which can be implemented on a Rogers TMM10 laminate board. 


The result of this thesis work is a rigorous technique that can be implemented to improve performance of transformation optics-based antennas as well as a novel fixed-beam leaky-wave antenna.






